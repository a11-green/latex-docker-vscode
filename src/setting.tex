% パッケージ
\usepackage[dvipdfmx]{hyperref, graphicx}
\usepackage{pxjahyper}
\usepackage{siunitx} % SI単位
\usepackage{amssymb}
\usepackage{amsmath}
\usepackage{caption,subcaption}
\usepackage[top=20truemm,bottom=20truemm,left=25truemm,right=25truemm]{geometry}
\usepackage{here}
\usepackage{ascmac} % 枠付き文章


\hypersetup{
    colorlinks=false, % リンクに色をつけない設定
    bookmarks=true, % 以下ブックマークに関する設定
    bookmarksnumbered=true,
    pdfborder={0 0 0},
    bookmarkstype=toc,
}

\newcommand{\linedhref}[2]{\underline{\href{#1}{\emph{#2}}}}

% 接頭辞の設定
\renewcommand{\figurename}{Fig.}
\renewcommand{\tablename}{Table.}
\newcommand\figref[1]{{図~\ref{fig:#1}}}
\newcommand\tabref[1]{{表~\ref{tab:#1}}}
\newcommand\eref[1]{{式~({\boldmath \ref{eq:#1}})}}
\newcommand\secref[1]{{第~\ref{sec:#1}章}}
\newcommand\ssecref[1]{{~\ref{sec:#1}節}}
\def\abstractname{要旨}
\def\prepartname{第}
\def\postpartname{章}

% 自作関数
\newcommand{\divergence}{\mathrm{div}\,}  %ダイバージェンス
\newcommand{\grad}{\mathrm{grad}\,}  %グラディエント
\newcommand{\rot}{\mathrm{rot}\,}  %ローテーション
\def\vector#1{\mbox{\boldmath $#1$}}
\newcommand{\e}{\mathrm{e}} %自然対数
\newcommand{\pdiff}[3]{
    \if 1#1 \dfrac{\partial #2}{\partial #3}
    \else \dfrac{\partial^{#1} #2}{\partial #3^{#1}}\fi
}

% Captionの設定
% \captionwidth=0.8\textwidth
\captionsetup[figure]{font=small}
\captionsetup[table]{font=small} %small=9pt,footnotesize=8pt}
\abovecaptionskip=5pt
\belowcaptionskip=0pt